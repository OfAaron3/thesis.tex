\chapter{Conclusions}\label{Chap:conclusion}

Through this work we have explored the use of statistical tools and radiative transfer modelling to attempt to understand prominences. We first introduced the example prominence used in this work. This was a very active and complex prominence classed, seemingly misleadingly as a quiescent prominence. As discussed in Chap. \ref{Chap:Intro}, the two main categories of active and quiescent do not mean that the latter is dormant or inactive. It is merely used to describe their location and predicted longevity. As we saw, this prominence continued to exist for days after the end of the main observation, eventually appearing in the 171~\AA{} and 304~\AA{} SECCHI images from STEREO-A, which is expected of prominences of this category. When the Doppler velocities were corrected for solar rotation by measuring the line core of the \mgiihk{} lines from the one slit containing the solar disc, and using these as the reference wavelengths, the prominence was found to be mainly dominated by redshifts. Around 10\% to 20\% of the line profiles in the prominence were classed as complex, which are found to be a problem for PROM in Chap. \ref{Chap:prom}. The creation and rationale behind the coronal filter I employ was also explained. It is designed to try to keep as much of the prominence material as possible, which is what necessitates the final filtering step. Sometimes it is difficult to deduce what is coronal and what is chromospheric on a purely statistical, pixel to pixel basis. 

The 1D modelling and inversions discussed in Chap. \ref{Chap:prom} using PROM are a very powerful tool to invert prominence atmospheres with. The 1D nature of PROM has its fair share of pros and cons. The cons are that it cannot simulate many threads along the line of sight easily, or employ some eccentric velocity fields like RTCY can from Chap. \ref{Chap:2DModel}. However, its biggest draw is its speed. Large grids of models can be created over the course of a few hours. xRMS, and its older sibling rRMS, were presented in this work. Unlike other algorithms xRMS matches the line profiles point-for-point in wavelength space. It does not first reduce them to a handful of statistical measurements. \cite{heinzel_understanding_2015} had done something similar but for the mean of the IRIS \mgiihk{} observation considered. One of the many benefits of xRMS is its speed, it can find the best fitting model, from some predetermined 1D model grid, for each and every pixel at a rate of around 115 models per minute on a dual socket motherboard with two Intel Xeon E5-2697 v4s. This quoted number is for a run over a single 32 step raster. Any change in raster size will cause the run time to change linearly. The additional speed comes from the parallelisation over the rasters, letting separate CPU threads handle separate rasters. If you have as many threads as rasters, the wall time will reflect the quoted single raster time. 

The obvious step forward with this modelling is to increase the size of the model grid. However, the run time scales linearly with the number of models, and this may become unmanageably slow without additional optimisation. One such optimisation could be to split the spectra and models into groups of similar spectral types, like \cite{panos_identifying_2018} did with IRIS solar flare spectra. With this, each of the groups of data can be matched only with models also in that group. This would greatly reduces the number of models matched to each spectra, but requires some preparation of the data beforehand. However, due to this increase in speed it would be tempting to increase the size of the model grid yet again. Therefore, continued optimisation of this grid search method would encourage the creation of larger and larger grids of models. This has already happened once with the increase from 1007 models to 23940 models between rRMS and xRMS. It would be more efficient (and precise) to design and implement some kind of Invertible Neural Network (INN), similar to that of RADYNVERSION by \cite{osborne_radynversion_2019}.

However, an unexplored issue with this method is the degeneracy in solution for 1D models, which is currently absent in the literature. This does not mean that 1D simulations should be discarded altogether. They are a good tool to use when starting to learn about radiative transfer, as they allow you to experiment with the plasma conditions in order to get a feel for in what way the line profiles are affected by different parameters. However, their inability to accurately model the complex geometry of reality highlights the need for higher dimensionality in our models. This should be motivation towards building an INN with 2D models instead of 1D. Of course, this has its own problems to overcome, mainly that reality is rarely as well behaved as simulations. We would require some way to implement transitions in temperature, pressure, and other parameters within the model itself. There is already a radiative transfer framework available which would allow us to do this, Lightweaver \citep{osborne_lightweaver_2021}. It has been used in a recently submitted paper by \cite{jenkins_full_2022} doing realistic 2D magnetohydrodynamic (MHD) simulations of solar filaments and their fine structure. It shows very promising results for the future of 2D prominence and filament modelling.

Currently in 2D modelling, there is the work of \cite{gouttebroze_radiative_2004,gouttebroze_radiative_2005,gouttebroze_radiative_2006,gouttebroze_radiative_2007,gouttebroze_radiative_2008,gouttebroze_radiative_2009}; and \cite{labrosse_radiative_2016}, RTCY. Through this, we investigated the formation of \mgiihk{} under different velocity modes. First we looked at the stationary mode to act as a reference for the non-stationary modes. In the stationary mode everything formed at rest wavelength as is expected. We then explored how Doppler velocity affects the line formation regions. Unsurprisingly it was the same as the stationary case, with everything shifted in the direction of the Doppler velocity. The most interesting case was the expanding case where the all the plasma in the cylinder was moving radially away from the axis of rotational symmetry. Here we saw very interesting Carlsson and Stein plots, where the contribution function displayed the complex way in which the line profile was formed. With the prominence moving at all radial velocities with some component towards the line-of-sight, this caused the line profiles to appear `round' in the spectrographs. This is due to the change in magnitude of the component of the velocity changing with the cosine of the angle between the line of sight and the velocity vector. We then moved on to show multithread simulations. The first simple unaligned stacked case showed that by the spectra alone it was not possible to discern the multithread configuration. However, it is possible to see there is some multithread structure by the integrated intensity over the slit. By introducing line of sight velocities, it is possible to produce asymmetric line profiles as in \cite{gunar_lyman-line_2008} and \cite{labrosse_radiative_2016}. To produce more striking asymmetry, like in the previously mentioned studies, larger velocities were required due to the nature of \mgiihk{} emission. When we convolve our final asymmetric profiles with the spatial and spectral point-spread-functions (PSFs) of the Interface Region Imaging Spectrograph \citep[IRIS; ][]{depontieu_interface_2014}, a lot of the interesting structure is destroyed, while still maintaining the asymmetry. Additionally this shows that the spatial PSF of IRIS could be removing interesting structure that we cannot recover through deconvolution as it is generally not a reversible process. 

An proof-of-concept 2D multithread inversion was presented. A similar type of multithread inversion had been performed by \cite{heinzel_understanding_2015}, but this was in 1D. With our generated 2D model, we selected two candidate locations from which our spectra could be matched -- one nearer to the top of the prominence where there would be less incident radiation, and one closer to the bottom where there would be more. Although these spectra come from the same model, the incident radiation is different, and this was the aim behind this decision. Both candidate locations were used to create the multithread generated spectra. These matches both fell well under the 15000 RMS threshold of xRMS and were therefore named as good matches. While this example was successful, this problem does not scale well computationally. A method to determine the number of threads present in some spectra would need to be implemented. This is not a trivial task to automate or perform manually. It may be well suited for some neural network architecture, but I am unaware of any suitable types of network. Even if these are found, using a tool like xRMS to compare the matches would be a slow endeavour. Every model from RTCY has 201 different positions in $z$, giving us 201 separate 1D spectra. Additionally, \cite{heinzel_formation_2014} reported on the importance of partial redistribution (PRD) to the formation of the \mgiihk{} resonance lines. RTCY, however, treats all of its resonance lines in complete redistribution (CRD). Additionally, it has recently by demonstrated by \cite{gunar_large_2022} that the solar cycle variability of the incident radiation plays an important role in the radiative transfer modelling of \mgiihk{}. The authors used the mean intensities of whole disc IRIS rasters as the incident radiation upon the prominence. The difference in simulated radiation between solar minimum and maximum was found to be up to 26\%! This work demonstrates the need to use location-specific incident radiation, as prominences can form in brighter or darker regions than the whole disc average. The radiation when and where the prominence exists should be used to accurately represent the conditions of the prominence. This was briefly explored by \cite{zhang_launch_2019}, but the impact this had on the simulations was not explored. In addition to the incident radiation from below the prominence, \cite{brown_influence_2018} modelled the influence of coronal radiation on solar prominences. While this applies mostly to Helium, this too changes with the solar cycle and should also be explored in more depth. Going forward, if we were to create some neural network or some other machine learning architecture to be applied to this problem, we should base it on a 2D code which includes PRD and location-specific incident intensity to accurately model the \mgiihk{} resonance lines. 

To conclude this work, we explored the movement of photospheric flows and their effect on the movement of magnetic flux patches and their associated phenomena. Fourier Local Correlation Tracking \citep[FLCT; ][]{welsch_ilct_2004,fisher_flct_2008} was used to determine the photospheric flows. Then, working under the assumption that these flows influence the movement of magnetic flux patches \citep{muglach_photospheric_2021} we worked to identify magnetic flux patches in data from the Helioseismic Imager \citep[HMI; ][]{scherrer_helioseismic_2012} pertaining to a coronal bright point (CBP) seen by the Atmospheric Imaging Assembly \citep[AIA; ][]{lemen_atmospheric_2012}, where both the former and latter instruments are onboard the Solar Dynamics Observatory \citep[SDO; ][]{pesnell_solar_2012}. These patches were identified through the use of a threshold and a connectivity filter. Any group of 2-connected pixels of flux magnitude greater than 30~G were identified as patches. From this, we were able to calculate the predicted bulk motion of the magnetic patches by finding the mean of all of the velocity vectors contained in the patch. From this we were able to reasonably predict the movement of these patches. The movement of these patches were seen to influence the stability of the minifilament at the polarity inversion line of the flux emergence. This minifilament was seen to erupt twice as flux patches began to undergo flux cancellation in the photosphere. The evolution of the photospheric magnetic field is very important to the stability of prominences. Further study should be undertaken to study the way in which the `magnetic hammock' in which prominences are supported evolves and becomes unstable. Spectropolarimetric observations of solar prominences would be a good place to begin, as the isolated nature of a prominence gives us a unique view of how this plasma is supported. Of course, we cannot exploit photospheric flows in this observing configuration, but this can be worked towards. Outside of prominences, \cite{potts_small-scale_2007} used local correlation tracking to track flows of magnetic patches in a similar manner to here to show that there exists small scale energy releases driven by these flows on the quiet Sun. They did this by correlating flux cancellation with soft X-ray emission seen with the Soft X-ray Telescope \citep[SXT; ][]{tsuneta_soft_1991} onboard Yohkoh \citep{ogawara_solar-mission_1991}. The authors go on to say that they wish to verify their findings using the X-ray Telescope \citep[XRT; ][]{golub_x-ray_2007} onboard Hinode \citep{kosugi_hinode_2007}. However, a follow up study is still to be conducted.

Looking further ahead, there are many exciting new opportunities for solar prominence observations in the future. With the launch of Solar Orbiter \citep[SolO; ][]{muller_solar_2020} we have access to a suite of new instruments performing `up close' observations of the Sun with a perihelion 60~R$_\odot$. This suite includes the Extreme Ultraviolet Imager \citep[EUI; ][]{rochus_solar_2020} with access to the first space first Ly~$\alpha$ imager since the Transition Region And Coronal Explorer \citep[TRACE; ][]{handy_transition_1999} returned its last science image in 2010. Additionally, the new Spectral Imaging of the Coronal Environment \citep[SPICE; ][]{spice_consortium_solar_2020} instrument onboard SolO will afford us new spectroscopic observations of Ly~$\beta$ and Ly~$\gamma$. We have not had spatially resolved spectroscopic observations of the Lyman lines since the Solar Ultraviolet Measurements of Emitted Radiation \citep[SUMER; ][]{wilhelm_sumer_1995} instrument on board the Solar and Heliospheric Observatory \citep[SOHO; ][]{domingo_soho_1995} went into permanent hibernation in the mid to late 2010s. Coordinated observations between SPICE, EUI, and IRIS would be very exciting. SPICE would allow us to resolve finer structure than is possible with IRIS and in the aforementioned Lyman lines. These many sets of lines could be used to better constrain and model solar prominences. Additionally, a new spectropolarimeter is now installed on the T\'{e}lescope Heliographique pour l'Etude du Magn\'{e}tisme et des Instabilit\'{e}s Solaires \citep[Heliographic Telescope for the Study of Solar Magnetism and Instabilities; THEMIS; ][]{mein_themis_1985} measuring the polarisation of the He D3 line, which can give us more insight into the magnetic structure of solar prominences. An ideal observation would be a coordination between all of these instruments -- there would be a treasure trove of coaligned data to explore.


The work presented in this thesis intends to demonstrate the power of the marriage between observation and modelling. We attempt to highlight the importance of comparing the line profiles from data to the line profiles generated by models point-for-point. The truth of the observation lies in the line profiles. Reducing them to a handful of statistics abstracts away the structure and information inherent in the spectra. We should not sacrifice information for speed. We also wished to show the power of modelling unique situations that give us further insight into the spectra that we see. The current state of prominence modelling is very exciting as we seem to be entering a new era of radiative transfer codes and inversion techniques. From this we will be able to understand more about the way in which our local star operates, and how we may apply this understanding to the wider field of stellar physics. I am excited by the opportunities afforded by the future of this field.